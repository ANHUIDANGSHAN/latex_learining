\documentclass[UTF8,12pt]{article}

\usepackage{ctex}
\usepackage{graph35}
\newcommand{\myfont}{\textsf{Fancy Text}}%新建一个命令,在正文中提醒
\title{\heiti 字体设置}

\begin{document}
	\maketitle
    (1) 字体簇设置(罗马字体、无衬线字体、打字机字体...
	
	\textrm{Roman Family}  \textsf{Sans Serif Family}
	\texttt{Typewiter Family}
	
	% 另一种设置方式
	{\rmfamily Roman Family}  {\sffamily Sans Serif Family}
	{\ttfamily Typewriter Family}
	
	{\sffamily who are you? you find self on every one around. take you as the same as others!}

	(2) 设置字体系列设置(粗细、宽度)
	
	\textmd{Medium Series}   \textbf{Boldface Series}
	
	{\mdseries Medium Series}    {\bfseries Boldface Series}
	
	(3) 设置字体形状(直立、斜体、伪斜体、小型大写)
	
	\textup{Upright Shape} \textit{Italic Shape} \textsl{Slanted Shape}  \textsc{Small Caps Shape}
	
	{\upshape Upright shape} {\itshape Italic Shape}
	{\slshape Slanted Shape} {\scshape Small Caps Shape}
	
	(4) 中文字体
	
	{\songti 宋体} \quad {\heiti 黑体} \quad {\fangsong 仿宋} \quad {\kaishu 楷书}
	
	中文字体中的\textbf{粗体}与\textit{斜体}
	
	(5) 字体大小
	
	{\tiny         Hello}
	
	{\scriptsize   Hello}\\
	{\footnotesize Hello}\\
	{\small        Hello}\\
	{\normalsize   Hello}\\
	{\large        Hello}\\
	{\Large        Hello}\\
	{\LARGE        Hello}\\
	{\huge         Hello}\\
	{\Huge         Hello}\\
	
	(6) 中文字号设置命令
	
	\zihao{5} 你好!%数字越大,字体越小,且命令的作用域为设置命令之后的所有文档
	
	\zihao{1} 你好!
	
	(7) 定义命令的使用
	
	\myfont
	
	%插入图片
	\begin{figure}[b]
	\centering
	\includegraphics[width=0.7\linewidth]{screenshot001}
	\caption{}
	\label{fig:screenshot001}
    \end{figure}
\end{document}



